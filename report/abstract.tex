% \begin{minipage}[t][0.49\textheight][t]{\textwidth}
\section*{Περίληψη}
\paragraph{} Αντικείμενο της παρούσας εργασίας ήταν η προσομοίωση της πρόσ\-πτω\-σης ενός
τσουνάμι σε ακτογραμμή πόλης. Έμφαση δόθηκε στη διατήρηση της ορμής, καθώς η κατανομή της
στο χώρο και το χρόνο είναι καθοριστική για τις επιπτώσεις του κύματος στην
ακτογραμμή. Για το λόγο αυτό, υιοθετήθηκε μια υβριδική μέθοδος προσομοίωσης, βασισμένη στη
μέθοδο \eng{Smoothed Particle Hydrodynamics} (\eng{SPH}), εμπλουτισμένη όμως με
γεωμετρικούς περιορισμούς και αλληλεπιδράσεις υλικών σωμάτων. Η υλοποίηση είναι αποτέλεσμα
της συνεργασίας της μηχανής φυσικής \eng{Bullet} και μιας μηχανής \eng{SPH}, που
επεξεργάζονται αλληλοδιάδοχα την δυναμική κατάσταση του ρευστού σε κάθε χρονικό
βήμα. Επιπλέον, για την καλύτερη απόδοση της προσομοίωσης αναπτύχθηκε ειδική δομή
δεδομένων (\eng{LP grid}) για τη διατήρηση της τοπικότητας κατά την αποθήκευση των
δεδομένων στη μνήμη και της ελαχιστοποίησης του χρόνου πρόσβασης σε αυτά. Τα δεδομένα της
προσομοίωσης εξάγονται σε ειδικής μορφής αρχεία κειμένου (\eng{VTK}), επιτρέποντας την εκ
των υστέρων διαδραστική οπτικοποίηση και επεξεργασία. Πειραματικά αποτελέσματα δείχνουν τα
πλεονεκτήματα της καταγραφής ώσεων για εκτίμηση πιθανής καταστροφής και την αξιολόγηση
στρατηγικών προστασίας.

\paragraph{Λέξεις Κλειδιά} Προσομοίωση ρευστών, τσουνάμι, \eng{SPH}, αλληλεπίδραση
τσουνάμι-α\-κτο\-γραμ\-μής, οπτικοποίηση δυνάμεων
% \end{minipage}
\clearpage
% \begin{minipage}[b][0.49\textheight][t]{\textwidth}
\selectlanguage{english}
\section*{Abstract}
\paragraph{} The objective of this thesis was the simulation of a tsunami impact upon an
urban coastline. Emphasis was given to the conservation of momentum, as its distribution
in space and time is the main factor of the wave's effects on the coastline. Due to this,
a hybrid simulation method was adopted, based on the Smoothed Particle Hydrodynamics (SPH)
method, enriched with geometric constraints and rigid body interactions. The
implementation is the result of cooperation between the Bullet physics engine and our
custom SPH engine, which successively process the dynamic state of the fluid at every
timestep. Furthermore, in order to achieve better performance a custom data structure (LP
grid) was developed for the optimization of locality in data storage and minimization of
access time. Simulation data is exported to VTK files, allowing interactive processing and
visualization. Experimental results demonstrate the benefits of impulse recording at
potential hazard estimation and evaluation of defense strategies.

\paragraph{Keywords} Fluid simulation, tsunami, SPH, tsunami-coastline interaction, force
visualization

\selectlanguage{greek}
% \end{minipage}
%%% Local Variables:
%%% mode: latex
%%% TeX-master: "report"
%%% End:
