% \begin{minipage}[t][0.49\textheight][t]{\textwidth}
\section*{Περίληψη}
\paragraph{} Αντικείμενο της παρούσας εργασίας ήταν η ακριβής προσομοίωση της
πρόσ\-πτω\-σης ενός τσουνάμι σε ακτογραμμή. Ιδιαίτερη έμφαση δόθηκε στη διατήρηση της
ορμής, καθώς η κατανομή της στο χώρο και το χρόνο είναι καθοριστική για τις επιπτώσεις του
κύματος στην ακτογραμμή. Για το λόγο αυτό, υιοθετήθηκε μια υβριδική μέθοδος προσομοίωσης,
βασισμένη στη μέθοδο \eng{SPH} (\eng{Smoothed Particle Hydrodynamics}), εμπλουτισμένη όμως
με γεωμετρικούς περιορισμούς και αλληλεπιδράσεις υλικών σωμάτων. Η υλοποίηση βασίζεται στη
συνεργασία της μηχανής φυσικής \eng{Bullet} και της μηχανής \eng{SPH}, που επεξεργάζονται
αλληλοδιάδοχα την δυναμική κατάσταση του ρευστού σε κάθε βήμα της προσομοίωσης. Για την
καλύτερη απόδοση της προσομοίωσης αναπτύχθηκε επιπλέον ειδική δομή δεδομένων (\eng{LP
  grid}) για την βέλτιστοποίηση της τοπικότητας κατά την αποθήκευση των δεδομένων στη
μνήμη και της ταχύτητας πρόσβασης σε αυτά. Τα δεδομένα της προσομοίωσης εξάγονται σε
ειδικής μορφής αρχεία κειμένου (\eng{VTK}) με σκοπό την εκ των υστέρων διαδραστική
οπτικοποίηση και επεξεργασία μέσω κατάλληλων προγραμμάτων (όπως του \eng{ParaView}).

\paragraph{Λέξεις Κλειδιά} Προσομοίωση ρευστών, τσουνάμι, \eng{SPH}, αλληλεπίδραση
τσουνάμι-α\-κτο\-γραμ\-μής, οπτικοποίηση δυνάμεων
% \end{minipage}
\cleardoublepage
% \begin{minipage}[b][0.49\textheight][t]{\textwidth}
\selectlanguage{english}
\section*{Abstract}
\paragraph{} The objective of this thesis was the accurate simulation of a tsunami hit on
a coastline. Emphasis was given to the conservation of momentum during the simulation, as
its distribution in space and time is the main factor of the wave's effects on the
coastline. Due to this, a hybrid simulation method was adopted, based on the SPH method
(Smoothed Particle Hydrodynamics), enriched with geometric constraints and rigid body
interactions. The implementation is the result of cooperation between the Bullet physics
engine and the custom SPH engine, which successively process the dynamic state of the
fluid at every timestep of the simulation. Furthermore, in order to achieve better
performance a custom data structure (LP grid) was developed for the optimization of
locality in data storage and minimization of access time. Simulation data is exported to
text files in VTK format to allow interactive processing and visualization with the aid of
specialized programs (like ParaView).
\paragraph{Keywords} Fluid simulation, tsunami, SPH, tsunami-coastline interaction,
force visualization

\selectlanguage{greek}
% \end{minipage}
%%% Local Variables:
%%% mode: latex
%%% TeX-master: "report"
%%% End:
